% Options for packages loaded elsewhere
\PassOptionsToPackage{unicode}{hyperref}
\PassOptionsToPackage{hyphens}{url}
%
\documentclass[
]{article}
\usepackage{amsmath,amssymb}
\usepackage{iftex}
\ifPDFTeX
  \usepackage[T1]{fontenc}
  \usepackage[utf8]{inputenc}
  \usepackage{textcomp} % provide euro and other symbols
\else % if luatex or xetex
  \usepackage{unicode-math} % this also loads fontspec
  \defaultfontfeatures{Scale=MatchLowercase}
  \defaultfontfeatures[\rmfamily]{Ligatures=TeX,Scale=1}
\fi
\usepackage{lmodern}
\ifPDFTeX\else
  % xetex/luatex font selection
\fi
% Use upquote if available, for straight quotes in verbatim environments
\IfFileExists{upquote.sty}{\usepackage{upquote}}{}
\IfFileExists{microtype.sty}{% use microtype if available
  \usepackage[]{microtype}
  \UseMicrotypeSet[protrusion]{basicmath} % disable protrusion for tt fonts
}{}
\makeatletter
\@ifundefined{KOMAClassName}{% if non-KOMA class
  \IfFileExists{parskip.sty}{%
    \usepackage{parskip}
  }{% else
    \setlength{\parindent}{0pt}
    \setlength{\parskip}{6pt plus 2pt minus 1pt}}
}{% if KOMA class
  \KOMAoptions{parskip=half}}
\makeatother
\usepackage{xcolor}
\usepackage[margin=1in]{geometry}
\usepackage{color}
\usepackage{fancyvrb}
\newcommand{\VerbBar}{|}
\newcommand{\VERB}{\Verb[commandchars=\\\{\}]}
\DefineVerbatimEnvironment{Highlighting}{Verbatim}{commandchars=\\\{\}}
% Add ',fontsize=\small' for more characters per line
\usepackage{framed}
\definecolor{shadecolor}{RGB}{248,248,248}
\newenvironment{Shaded}{\begin{snugshade}}{\end{snugshade}}
\newcommand{\AlertTok}[1]{\textcolor[rgb]{0.94,0.16,0.16}{#1}}
\newcommand{\AnnotationTok}[1]{\textcolor[rgb]{0.56,0.35,0.01}{\textbf{\textit{#1}}}}
\newcommand{\AttributeTok}[1]{\textcolor[rgb]{0.13,0.29,0.53}{#1}}
\newcommand{\BaseNTok}[1]{\textcolor[rgb]{0.00,0.00,0.81}{#1}}
\newcommand{\BuiltInTok}[1]{#1}
\newcommand{\CharTok}[1]{\textcolor[rgb]{0.31,0.60,0.02}{#1}}
\newcommand{\CommentTok}[1]{\textcolor[rgb]{0.56,0.35,0.01}{\textit{#1}}}
\newcommand{\CommentVarTok}[1]{\textcolor[rgb]{0.56,0.35,0.01}{\textbf{\textit{#1}}}}
\newcommand{\ConstantTok}[1]{\textcolor[rgb]{0.56,0.35,0.01}{#1}}
\newcommand{\ControlFlowTok}[1]{\textcolor[rgb]{0.13,0.29,0.53}{\textbf{#1}}}
\newcommand{\DataTypeTok}[1]{\textcolor[rgb]{0.13,0.29,0.53}{#1}}
\newcommand{\DecValTok}[1]{\textcolor[rgb]{0.00,0.00,0.81}{#1}}
\newcommand{\DocumentationTok}[1]{\textcolor[rgb]{0.56,0.35,0.01}{\textbf{\textit{#1}}}}
\newcommand{\ErrorTok}[1]{\textcolor[rgb]{0.64,0.00,0.00}{\textbf{#1}}}
\newcommand{\ExtensionTok}[1]{#1}
\newcommand{\FloatTok}[1]{\textcolor[rgb]{0.00,0.00,0.81}{#1}}
\newcommand{\FunctionTok}[1]{\textcolor[rgb]{0.13,0.29,0.53}{\textbf{#1}}}
\newcommand{\ImportTok}[1]{#1}
\newcommand{\InformationTok}[1]{\textcolor[rgb]{0.56,0.35,0.01}{\textbf{\textit{#1}}}}
\newcommand{\KeywordTok}[1]{\textcolor[rgb]{0.13,0.29,0.53}{\textbf{#1}}}
\newcommand{\NormalTok}[1]{#1}
\newcommand{\OperatorTok}[1]{\textcolor[rgb]{0.81,0.36,0.00}{\textbf{#1}}}
\newcommand{\OtherTok}[1]{\textcolor[rgb]{0.56,0.35,0.01}{#1}}
\newcommand{\PreprocessorTok}[1]{\textcolor[rgb]{0.56,0.35,0.01}{\textit{#1}}}
\newcommand{\RegionMarkerTok}[1]{#1}
\newcommand{\SpecialCharTok}[1]{\textcolor[rgb]{0.81,0.36,0.00}{\textbf{#1}}}
\newcommand{\SpecialStringTok}[1]{\textcolor[rgb]{0.31,0.60,0.02}{#1}}
\newcommand{\StringTok}[1]{\textcolor[rgb]{0.31,0.60,0.02}{#1}}
\newcommand{\VariableTok}[1]{\textcolor[rgb]{0.00,0.00,0.00}{#1}}
\newcommand{\VerbatimStringTok}[1]{\textcolor[rgb]{0.31,0.60,0.02}{#1}}
\newcommand{\WarningTok}[1]{\textcolor[rgb]{0.56,0.35,0.01}{\textbf{\textit{#1}}}}
\usepackage{graphicx}
\makeatletter
\def\maxwidth{\ifdim\Gin@nat@width>\linewidth\linewidth\else\Gin@nat@width\fi}
\def\maxheight{\ifdim\Gin@nat@height>\textheight\textheight\else\Gin@nat@height\fi}
\makeatother
% Scale images if necessary, so that they will not overflow the page
% margins by default, and it is still possible to overwrite the defaults
% using explicit options in \includegraphics[width, height, ...]{}
\setkeys{Gin}{width=\maxwidth,height=\maxheight,keepaspectratio}
% Set default figure placement to htbp
\makeatletter
\def\fps@figure{htbp}
\makeatother
\setlength{\emergencystretch}{3em} % prevent overfull lines
\providecommand{\tightlist}{%
  \setlength{\itemsep}{0pt}\setlength{\parskip}{0pt}}
\setcounter{secnumdepth}{5}
\usepackage{fancyhdr}
\usepackage{caption}
\captionsetup[table]{justification=raggedright, singlelinecheck=false} % Left-justified table captions
\usepackage[margin=1in]{geometry} % Adjusts page margins to prevent text overflow
\setlength{\parindent}{0pt} % Removes indentation on the title page for alignment
\pagenumbering{gobble} % Hides page number on title page
\usepackage{booktabs}
\usepackage{longtable}
\usepackage{array}
\usepackage{multirow}
\usepackage{wrapfig}
\usepackage{float}
\usepackage{colortbl}
\usepackage{pdflscape}
\usepackage{tabu}
\usepackage{threeparttable}
\usepackage{threeparttablex}
\usepackage[normalem]{ulem}
\usepackage{makecell}
\usepackage{xcolor}
\ifLuaTeX
  \usepackage{selnolig}  % disable illegal ligatures
\fi
\IfFileExists{bookmark.sty}{\usepackage{bookmark}}{\usepackage{hyperref}}
\IfFileExists{xurl.sty}{\usepackage{xurl}}{} % add URL line breaks if available
\urlstyle{same}
\hypersetup{
  pdftitle={``Navigating the Digital Divide: The Role of ICT in Enhancing Educational Outcomes in Thailand''},
  hidelinks,
  pdfcreator={LaTeX via pandoc}}

\title{``Navigating the Digital Divide: The Role of ICT in Enhancing
Educational Outcomes in Thailand''}
\author{Leigh Pearson, Ph.D.~Candidate\\
International College, Mahidol University\\
School of Global Studies, Thammasat University\\
\href{mailto:leigh.pea@mahidol.edu}{\nolinkurl{leigh.pea@mahidol.edu}}}
\date{Last Updated: 17-11-2024}

\begin{document}
\maketitle

\textbf{Note}: The analyses and findings presented in this document are
part of an ongoing PhD research project focused on ICT use in education
within Thailand. These results represent preliminary findings and are
intended to fulfill initial publication requirements. The complete
research project, which includes further analyses and comprehensive
conclusions, is described in full within the
\href{https://github.com/justleigh/PhD-ICT-Education-Analysis}{PhD-ICT-Education-Analysis
repository}. As the project progresses, additional insights and updates
will be incorporated to provide a complete view of ICT's role in
educational outcomes.

\pagenumbering{arabic}

\newpage
\tableofcontents

\newpage

\hypertarget{introduction}{%
\section{Introduction}\label{introduction}}

In an era where Information and Communications Technology (ICT) has
become a cornerstone of global development, the role of ICT in education
is especially crucial. Countries around the world are leveraging digital
technologies to enhance learning, improve equity, and drive innovation
in education.

The Thai education system has been a topic of concern for policymakers,
educators, and parents for many years, and yet Thailand has made
significant, often under-appreciated progress towards realising its
aspiration of transforming into a high-income, knowledge-based economy.
The country's vision details ambitious goals to upgrade the quality of
education and provide its young people with 21st-century skills, as laid
out by the National Strategy Committee (NSC, 2019) in its twenty-year
National Strategy (2017--2036) and by the Office of the National
Economic and Social Development Council (ONESDC, 2022) in its 12th and
13th National Economic and Social Development Plan (2017-2021 \&
2023-2027).

Despite these efforts, Thai students' performance on international
benchmarks has consistently lagged behind regional and global peers. For
instance, the average scores for the five O-NET subjects have
consistently fallen below 50 percent (National Institute of Educational
Testing Service, 2021), PISA scores have been in an increasingly steep
decline since 2012 in all three skills, particularly reading (OECD,
2023), and Thailand ranks 101st out of 113 participating countries for
English proficiency (English First, 2023).

This discrepancy points to a critical gap between policy aspirations and
on-the-ground realities. While there has been substantial progress in
terms of ICT infrastructure and resources, the impact of these
advancements on actual learning outcomes remains unclear. Furthermore,
disparities in ICT access between urban and rural students, known as the
`first' digital divide, along with the `second' digital divide, which
reflects inequalities in ICT usage, together intensify educational
inequalities, particularly affecting marginalized and disadvantaged
students (Ma \& Cheng, 2022).

This research, situated at the intersection of national policy and
educational outcomes, aims to explore how ICT integration within Thai
schools influences learning outcomes and equity. Using a combination of
the PISA ICT Framework (OECD, 2019) and the Multi-Level Framework of
Technology Acceptance and Use (MLFTAU; Venkatesh et al., 2016), this
study will provide a nuanced understanding of both structural and
behavioral factors that shape ICT use in educational settings.

Through an analysis of PISA datasets spanning multiple years and
complemented by data from key national agencies, such as the National
Statistical Office of Thailand, the Equitable Education Fund (EEF), and
the Office of Basic Education Commission (OBEC), this research will
assess the effectiveness of ICT in enhancing educational outcomes.
Special attention will be given to how ICT impacts underprivileged
students, who are often left behind in national efforts to digitize
education.

By examining Thailand's experience, this research will contribute
valuable insights to the global discourse on the role of technology in
education, particularly in developing countries facing similar digital
and educational challenges.

\hypertarget{utilization-of-conceptual-and-theoretical-frameworks}{%
\section{Utilization of Conceptual and Theoretical
Frameworks}\label{utilization-of-conceptual-and-theoretical-frameworks}}

This study utilizes two primary frameworks---the PISA ICT Framework and
the Multi-Level Framework of Technology Acceptance and Use (MLFTAU).
These frameworks were chosen due to their complementary strengths in
addressing both systemic and behavioral aspects of ICT integration,
aligning with the study's goals to evaluate how ICT use influences
educational outcomes in Thailand.

\hypertarget{pisa-ict-framework}{%
\subsection{PISA ICT Framework}\label{pisa-ict-framework}}

The PISA ICT Framework was developed by the OECD to specifically assess
the impact of Information and Communication Technology (ICT) within
educational systems across countries. This framework is adapted from the
older CIPO model, which is structured around four components:

\textbf{- Context:} The broader environment in which the educational
system operates, including socio-economic, policy, and infrastructure
factors that shape ICT access and use.

\textbf{- Input:} Resources and support systems provided to schools,
such as funding, ICT infrastructure, and training for teachers, as well
as the personal characteristics of students and teachers.

\textbf{- Process:} The methods, pedagogies, and practices through which
ICT is implemented within educational settings, both in-class and
outside-the-classroom.

\textbf{- Output:} The measurable outcomes of ICT use, including
academic performance, student engagement, and digital skills.

In the PISA ICT Framework, these four elements are adapted to reflect
the specific dynamics of ICT usage in education. It provides a structure
for analyzing both in-school and outside-the-classroom factors, such as
the availability of ICT resources, policies supporting digital
education, and the impact on students' performance and engagement.

The PISA ICT Framework was chosen for this study for the following
reasons:

\textbf{Systemic Analysis of ICT Integration}

The PISA ICT Framework retains the CIPO model's structured approach to
analyzing educational systems by dividing ICT impact into in-school
(in-class) and outside-the-classroom components. This layered approach
enables a comprehensive assessment of both the resources available and
the contextual factors that influence ICT use, providing a clear
structure for understanding ICT's role across various educational
settings.

\textbf{Alignment with International Benchmarks}

The PISA ICT Framework is tailored to evaluate ICT's impact within the
context of PISA's global educational assessments, directly supporting
the study's focus on assessing ICT-related educational outcomes in
Thailand. By applying a framework already validated within the PISA
system, this research can leverage internationally recognized metrics
and benchmarks to evaluate ICT's effectiveness in enhancing learning.

\textbf{Outcome-Focused Structure}

The PISA ICT Framework is outcome-driven, making it suitable for
measuring tangible educational results like student performance on PISA
benchmarks. This outcome orientation aligns directly with the study's
aim to evaluate how ICT usage impacts academic performance, engagement,
and equity in Thailand's educational system.

\hypertarget{multi-level-framework-of-technology-acceptance-and-use-mlftau}{%
\subsection{Multi-Level Framework of Technology Acceptance and Use
(MLFTAU)}\label{multi-level-framework-of-technology-acceptance-and-use-mlftau}}

The Multi-Level Framework of Technology Acceptance and Use (MLFTAU),
developed by Venkatesh and colleagues, provides a behavioral and
psychological perspective on the adoption and effective use of
technology. This framework considers factors at multiple
levels---individual, organizational, and social---that influence how
technology is perceived and utilized. The key components of MLFTAU
include:

\textbf{- Individual Level:} Focuses on personal factors that influence
technology use, such as perceived usefulness, perceived ease of use, and
behavioral intention. These factors impact whether individuals (students
and teachers) are willing to use ICT tools.

\textbf{- Organizational Level:} Examines how institutional support,
resources, and training affect technology adoption within organizations
(e.g., schools). This level is important for understanding how school
resources and policies influence ICT integration.

\textbf{- Social Influence:} Considers the impact of peer influence,
societal expectations, and cultural norms on technology use. For
example, students may be influenced by how their peers or teachers view
and use ICT.

\textbf{- Facilitating Conditions:} Reflects the infrastructure,
support, and resources that enable or constrain effective ICT use. This
includes both material resources (such as access to computers) and
institutional support.

The MLFTAU framework was selected for this study for the following
reasons:

\textbf{Focus on Behavioral Drivers}

MLFTAU captures individual attitudes, beliefs, and behavioral intentions
related to technology use among teachers and students. By examining
constructs like performance expectancy, effort expectancy, social
influence, and facilitating conditions, this framework allows the study
to investigate the psychological factors that drive or hinder ICT
adoption in educational settings.

\textbf{Applicability to Educational Stakeholders}

MLFTAU's emphasis on understanding user acceptance of technology is
especially relevant to this study's focus on teachers' and students' ICT
usage. Given that effective ICT integration relies on both access and
willingness to use technology, MLFTAU adds depth to the analysis by
examining how individual motivations and institutional support affect
ICT engagement.

\textbf{Complementary to Systemic Analysis}

While the PISA ICT Framework provides a structured view of ICT's
systemic impact on educational outcomes, MLFTAU offers a granular
understanding of the behavioral factors influencing ICT adoption. This
dual approach captures both the structural and personal dimensions of
ICT use, offering a holistic view of its role in Thailand's educational
landscape.

In summary, the combined use of the PISA ICT Framework and MLFTAU
enables a comprehensive analysis that spans both external, systemic
factors and individual behavioral drivers. This integration aligns with
the study's objectives to evaluate ICT's impact on educational outcomes
while addressing the practical and psychological barriers to effective
ICT use in Thai schools.

\hypertarget{application-and-integration-of-frameworks-in-data-analysis}{%
\subsection{Application and Integration of Frameworks in Data
Analysis}\label{application-and-integration-of-frameworks-in-data-analysis}}

This study applies the CIPO and MLFTAU frameworks synergistically to
examine the systemic and behavioral factors influencing ICT use in
education. The CIPO model structures the investigation into how
contextual, input, and process variables affect student outcomes, while
the MLFTAU framework adds depth by analyzing individual attitudes and
behaviors toward ICT acceptance within schools. This dual-framework
approach enables a nuanced analysis that captures both the external
factors shaping ICT access and the behavioral factors that drive ICT
use.

In this analysis, data from multiple sources---including the PISA
student and ICT questionnaires, along with national datasets like the
EMIS and NSO---will be mapped onto the PISA ICT framework, which aligns
with the CIPO model's dimensions of input, process, and output.
Contextual information from national datasets provides insights into
socio-economic and regional disparities, while process variables, such
as ICT integration in classrooms, are explored using teacher and student
responses from PISA questionnaires.

The MLFTAU framework adds a behavioral perspective to the analysis of
ICT use. For example, regression models will assess how factors such as
performance expectancy and facilitating conditions predict ICT usage
among teachers and students, thereby influencing educational outcomes.
This combined approach captures the systemic and behavioral complexities
of ICT use, aligning with the research objectives of assessing ICT
integration, identifying barriers and enablers, and making informed
policy recommendations.

\hypertarget{objectives}{%
\section{Objectives}\label{objectives}}

The objectives of this research are to:

\textbf{1. Assess the current state of ICT integration in Thai education
and its alignment with national strategy aspirations.}

By understanding how closely current ICT practices align with Thailand's
strategic goals, this objective provides insights into whether
investments and policy directives are translating effectively into
school environments. This analysis will reveal gaps in ICT resources or
infrastructure that may hinder educational quality, particularly in
digital literacy and access.

The findings will help policymakers identify where the strategy falls
short in practical application, enabling targeted adjustments. For
example, if schools lack sufficient devices or broadband access, the
government can prioritize these resources in underperforming areas,
thereby supporting equitable learning environments and improving digital
competencies across regions.

\textbf{2. Identify the enablers and barriers to effective ICT use
within educational settings.}

Identifying the factors that encourage or obstruct ICT use in classrooms
(such as teacher training, ICT resources, or administrative support)
directly impacts student learning experiences. Schools with adequate
enablers, like well-trained teachers and high-quality ICT
infrastructure, are likely to see enhanced student engagement and better
digital skill acquisition.

By pinpointing specific barriers, policymakers can implement targeted
initiatives to address them, such as expanding teacher training programs
or developing supportive policies. Removing these barriers supports more
effective technology integration, enabling students to benefit from ICT
resources and enhancing their overall learning outcomes, especially in
digital skills essential for future job markets.

\textbf{3. Analyze the relationship between ICT utilization and student
performance on established educational benchmarks, using Plausible
Values (PVs) and appropriate statistical models.}

This objective explores how ICT usage correlates with academic
performance across different socio-economic and geographic backgrounds,
highlighting the role of ICT in bridging or widening educational
inequalities. By examining this relationship, the study can determine
whether ICT access is genuinely beneficial to student achievement or if
disparities in usage contribute to unequal outcomes.

The results can guide policies aimed at reducing educational
inequalities by improving ICT access for underserved groups, such as
students in rural areas or from lower socioeconomic backgrounds. For
instance, if ICT access is found to improve PISA performance, targeted
support for disadvantaged students could be a powerful equalizer,
potentially enhancing overall national performance on educational
benchmarks.

\textbf{4. Provide evidence-based recommendations to policymakers for
enhancing ICT acceptance and utilization in Thai education.}

Evidence-based recommendations can drive sustainable improvements in ICT
use across Thai schools, addressing both systemic (CIPO model) and
behavioral (MLFTAU) factors. These recommendations help ensure that the
integration of ICT not only aligns with national goals but also supports
enhanced educational equity and outcomes.

This objective synthesizes findings into actionable guidance, such as
recommending specific infrastructure improvements, training initiatives,
or policies that promote equitable ICT access. It can also propose
frameworks for ongoing assessment and adjustment, ensuring that ICT
policies remain responsive to emerging challenges and changing
educational needs.

To clarify how each research objective aligns with the study's
analytical frameworks, the following table maps each objective to the
relevant dimensions of the CIPO model---Context, Input, Process, and
Output---and the Multi-Level Framework of Technology Acceptance and Use
(MLFTAU). Additionally, it distinguishes between in-school and
outside-the-classroom contexts, reflecting the dual environment in which
ICT influences student learning. This multi-layered approach underscores
the study's comprehensive analysis of ICT integration by examining both
structural factors within educational systems and behavioral factors
that affect individual technology acceptance. By situating each
objective within this combined framework, the table provides a visual
guide to the study's scope, illustrating how it addresses ICT's role in
enhancing educational outcomes in Thailand across different settings.

\begin{longtable}[t]{>{\raggedright\arraybackslash}p{3cm}>{\raggedright\arraybackslash}p{2.5cm}>{\raggedright\arraybackslash}p{2.5cm}>{\raggedright\arraybackslash}p{7cm}}
\caption{\label{tab:mapping_objectives_to_CIPO_and_MLFTAU_frameworks}Mapping Research Objectives to CIPO and MLFTAU Framework Dimensions}\\
\toprule
Objective & CIPO Framework \textbackslash{} Dimension & MLFTAU \textbackslash{} Dimension & Explanation\\
\midrule
1. Assess the current state of ICT integration in Thai education and its alignment with national strategy aspirations. & Context, Input & Facilitating Conditions & Context: Evaluates external factors, such as socio-economic status and infrastructure availability, affecting ICT integration in and out of school.
Input: Assesses resources (e.g., funding, technology, teacher training) provided to implement ICT in schools and its availability at home.
Facilitating Conditions: Examines if infrastructure and support are in place to enable ICT adoption in both settings.\\
2. Identify the enablers and barriers to effective ICT use within educational settings. & Input, Process & Effort Expectancy, Social Influence, Facilitating Conditions & Input: Identifies resources and training available for ICT adoption in school.
Process: Analyzes classroom ICT usage practices and obstacles.
Effort Expectancy: Considers ease of ICT use for educators and students within the school.
Social Influence: Examines peer or policy pressures to adopt ICT in classrooms.
Facilitating Conditions: Identifies existing school-based support for effective ICT use.\\
3. Analyze the relationship between ICT utilization and student performance on established educational benchmarks, using Plausible Values (PVs) and appropriate statistical models. & Process, Output & Performance Expectancy & Process: Investigates ICT integration into teaching practices in school and the use of ICT for homework or self-directed learning at home.
Output: Assesses the impact of ICT on academic performance both in-school (e.g., classroom engagement) and outside-the-classroom (e.g., homework effectiveness).
Performance Expectancy: Analyzes if ICT is perceived to improve educational outcomes in both settings.\\
4. Provide evidence-based recommendations to policymakers for enhancing ICT acceptance and utilization in Thai education. & All (Context, Input, Process, Output) & All MLFTAU Dimensions & Context: Considers socio-economic and regional factors affecting ICT access in both school and home environments.
Input: Recommends necessary resources and training in schools and for home access.
Process: Suggests best practices for ICT integration in classrooms and strategies for encouraging productive ICT use at home.
Output: Aims to improve student performance and equity through enhanced ICT policies across both in-school and outside-the-classroom environments.
MLFTAU Dimensions: Provides recommendations to address all behavioral and structural factors affecting ICT adoption and effective u\\
\bottomrule
\end{longtable}

The potential of this research extends beyond national boundaries,
echoing the global challenges of digital technology integration within
educational systems. By delineating Thailand's experience, insights of
broader relevance are revealed, contributing to the international
discourse on educational technology and its implementation in diverse
cultural and economic contexts.

\hypertarget{methodology}{%
\section{Methodology}\label{methodology}}

This research employs a quantitative approach, leveraging secondary data
from sources including the PISA assessments and national statistics to
analyze the role of Information and Communications Technology (ICT) in
shaping educational outcomes in Thailand. Two guiding frameworks are
central to this study: the PISA ICT Framework, which is based on the
CIPO model (Context, Input, Process, Output), and the Multi-Level
Framework of Technology Acceptance and Use (MLFTAU).

The PISA ICT Framework provides a structural perspective on ICT
integration within educational practices, examining contextual, input,
process, and output factors. In contrast, the MLFTAU offers insights
into the behavioral and psychological factors influencing ICT acceptance
and usage among students and educators. Together, these frameworks allow
for a comprehensive analysis of both systemic and individual factors
affecting ICT's impact on education.

To comprehensively assess the impact of ICT on student performance, this
study focuses on core variables such as ICT access, usage patterns, and
educational outcomes. Analytical techniques, including regression models
and hierarchical linear modeling, will be applied to examine these
relationships, with Plausible Values (PVs) from PISA data utilized to
account for measurement uncertainty in student performance. By
addressing both systemic and individual dimensions, this methodology
aims to generate nuanced and actionable insights, informing policymakers
on effective strategies to enhance ICT integration and reduce
educational disparities.

\hypertarget{data-sources}{%
\subsection{Data Sources}\label{data-sources}}

To conduct this analysis, the study leverages several key datasets.

Programme for International Student Assessment (PISA): Data from the
PISA assessments for 2000, 2003, 2006, 2009, 2012, 2015, 2018, and 2022
will be utilized. These datasets provide student performance data in
reading, mathematics, science, and creative thinking, as well as
contextual information from student and school questionnaires. The study
will focus on ICT-related variables as captured in the PISA surveys.

Thai National Educational Datasets: Additional data from the National
Statistical Office of Thailand, the Information System for Equitable
Education (iSEE 2.0), the Education Management Information System
(EMIS), and the Management Information System (MIS) of NIETS will
provide local context and complementary insights regarding ICT
infrastructure and usage.

These datasets collectively enable a robust analysis of ICT's impact on
educational outcomes across different contexts in Thailand. The PISA
dataset, in particular, includes Plausible Values (PVs) and weights,
ensuring that analysis reflects the complex sampling design used in the
assessment.

\hypertarget{data-import-and-setup}{%
\subsection{Data Import and Setup}\label{data-import-and-setup}}

The PISA data for analysis is imported using the intsvy package, which
is specifically recommended by the OECD for handling PISA data. This
package is tailored for large-scale international assessments, allowing
for the use of plausible values (PVs) and survey weights to account for
the complex sampling design employed in these assessments. The PISA
datasets used in this research include Thai-specific data from multiple
cycles (e.g., 2000, 2003, 2006, 2009, 2012, 2015, 2018, and 2022), along
with supplementary data from the National Statistical Office of Thailand
and other relevant sources. Proper data import and preparation are
essential to ensure that the analysis is based on clean, structured
data.

The original Thailand-specific PISA dataset from PISA Thailand is
provided in CSV format, which is compatible with intsvy and allows for
efficient processing without further conversion. The CSV files are
organized within the data/raw/ folder by cycle year, maintaining a clear
and organized version history for multi-year analysis.

The following libraries are essential for importing and handling the
PISA data and performing subsequent data manipulation:

\begin{Shaded}
\begin{Highlighting}[]
\FunctionTok{library}\NormalTok{(intsvy)     }\CommentTok{\# For handling PISA data analysis, plausible values, and survey weights}
\FunctionTok{library}\NormalTok{(readr)      }\CommentTok{\# For reading CSV files}
\FunctionTok{library}\NormalTok{(dplyr)      }\CommentTok{\# For data manipulation}
\FunctionTok{library}\NormalTok{(tidyr)      }\CommentTok{\# For tidying data}
\FunctionTok{library}\NormalTok{(ggplot2)    }\CommentTok{\# For data visualization}
\end{Highlighting}
\end{Shaded}

This code block loads all the required packages, ensuring that the
necessary tools for data handling and analysis are available.

To enable the loading of PISA data dynamically for each specified year,
a function is created to construct the file path based on the year
argument. This setup enables handling of data from multiple years
without the need for hardcoded file paths.

\begin{Shaded}
\begin{Highlighting}[]
\CommentTok{\# Function to dynamically load the PISA data for a specified year}
\NormalTok{load\_pisa\_data }\OtherTok{\textless{}{-}} \ControlFlowTok{function}\NormalTok{(year) \{}
    \CommentTok{\# Define file path dynamically}
\NormalTok{    csv\_path }\OtherTok{\textless{}{-}} \FunctionTok{paste0}\NormalTok{(}\StringTok{"data/raw/"}\NormalTok{, year, }\StringTok{"/pisa"}\NormalTok{, year, }\StringTok{"\_data.csv"}\NormalTok{)}
    \CommentTok{\# Read the CSV}
\NormalTok{    pisa\_data }\OtherTok{\textless{}{-}} \FunctionTok{read\_csv}\NormalTok{(csv\_path)}
    \FunctionTok{return}\NormalTok{(pisa\_data)}
\NormalTok{\}}
\end{Highlighting}
\end{Shaded}

To facilitate multi-year analysis, lists are created to store plausible
values (PVs) and weight variables specific to each year. The example
pv\_list below contains the PVs for mathematics performance across
different years, while the weights list holds the corresponding survey
weights. This approach allows the code to dynamically select the
appropriate variables based on the year being analyzed, ensuring
flexibility and scalability.

\begin{Shaded}
\begin{Highlighting}[]
\CommentTok{\# Example lists for PVs and weights by year, to facilitate multi{-}year analysis}
\NormalTok{pv\_list }\OtherTok{\textless{}{-}} \FunctionTok{list}\NormalTok{(}\StringTok{"2022"} \OtherTok{=} \FunctionTok{c}\NormalTok{(}\StringTok{"PV1MATH"}\NormalTok{, }\StringTok{"PV2MATH"}\NormalTok{, }\StringTok{"PV3MATH"}\NormalTok{, }\StringTok{"PV4MATH"}\NormalTok{, }\StringTok{"PV5MATH"}\NormalTok{),}
                \StringTok{"2018"} \OtherTok{=} \FunctionTok{c}\NormalTok{(}\StringTok{"PV1MATH18"}\NormalTok{, }\StringTok{"PV2MATH18"}\NormalTok{, }\StringTok{"PV3MATH18"}\NormalTok{, }\StringTok{"PV4MATH18"}\NormalTok{, }\StringTok{"PV5MATH18"}\NormalTok{))}
\NormalTok{weights }\OtherTok{\textless{}{-}} \FunctionTok{list}\NormalTok{(}\StringTok{"2022"} \OtherTok{=} \StringTok{"W\_FSTUWT"}\NormalTok{, }\StringTok{"2018"} \OtherTok{=} \StringTok{"W\_FSTUWT18"}\NormalTok{)}
\end{Highlighting}
\end{Shaded}

To validate the successful import and basic structure of each dataset, a
structural and summary analysis was conducted. This analysis included
generating a summary and structure file for each dataset year, allowing
a comprehensive examination of the dataset's layout and missing values.
Due to the extensive nature of this output, the complete results are
stored as separate files in the data/metadata/data\_exploration/
subfolder, organized by year. This ensures that the RMarkdown document
remains focused, while all necessary metadata files are accessible for
reference. The following code provides an example of this structure
generation process:

\begin{Shaded}
\begin{Highlighting}[]
\CommentTok{\# Example: Generate structure and summary for 2022 and save in metadata folder}
\NormalTok{year }\OtherTok{\textless{}{-}} \StringTok{"2022"}
\NormalTok{pisa\_data }\OtherTok{\textless{}{-}} \FunctionTok{load\_pisa\_data}\NormalTok{(year)}

\CommentTok{\# Save structure as a .txt file}
\NormalTok{structure\_file }\OtherTok{\textless{}{-}} \FunctionTok{paste0}\NormalTok{(}\StringTok{"data/metadata/data\_exploration/"}\NormalTok{, year, }\StringTok{"\_structure.txt"}\NormalTok{)}
\FunctionTok{capture.output}\NormalTok{(}\FunctionTok{str}\NormalTok{(pisa\_data), }\AttributeTok{file =}\NormalTok{ structure\_file)}

\CommentTok{\# Save summary as a .csv file}
\NormalTok{summary\_file }\OtherTok{\textless{}{-}} \FunctionTok{paste0}\NormalTok{(}\StringTok{"data/metadata/data\_exploration/"}\NormalTok{, year, }\StringTok{"\_summary.csv"}\NormalTok{)}
\FunctionTok{write.csv}\NormalTok{(}\FunctionTok{summary}\NormalTok{(pisa\_data), summary\_file)}
\end{Highlighting}
\end{Shaded}

In the next step, data for a specific year (in this example, 2022) is
loaded using the load\_pisa\_data function. The gender variable
(ST004D01T) is identified and converted to a factor with labels
(``Female'' and ``Male'') to facilitate analysis by gender. To confirm
the successful processing of this variable, a count of the number of
males and females is displayed. This serves as an initial data
validation step, ensuring that the gender data is correctly labeled and
ready for analysis.

\begin{Shaded}
\begin{Highlighting}[]
\CommentTok{\# Load data for a specific year (2022) and validate gender data}
\NormalTok{year }\OtherTok{\textless{}{-}} \StringTok{"2022"}
\NormalTok{pisa\_data }\OtherTok{\textless{}{-}} \FunctionTok{load\_pisa\_data}\NormalTok{(year)}

\CommentTok{\# Convert gender variable (ST004D01T) to a factor with labels}
\ControlFlowTok{if}\NormalTok{ (}\StringTok{"ST004D01T"} \SpecialCharTok{\%in\%} \FunctionTok{colnames}\NormalTok{(pisa\_data)) \{}
\NormalTok{    pisa\_data}\SpecialCharTok{$}\NormalTok{gender }\OtherTok{\textless{}{-}} \FunctionTok{factor}\NormalTok{(pisa\_data}\SpecialCharTok{$}\NormalTok{ST004D01T, }\AttributeTok{levels =} \FunctionTok{c}\NormalTok{(}\DecValTok{1}\NormalTok{, }\DecValTok{2}\NormalTok{), }\AttributeTok{labels =} \FunctionTok{c}\NormalTok{(}\StringTok{"Female"}\NormalTok{, }\StringTok{"Male"}\NormalTok{))}
\NormalTok{\} }\ControlFlowTok{else}\NormalTok{ \{}
    \FunctionTok{stop}\NormalTok{(}\StringTok{"The \textquotesingle{}ST004D01T\textquotesingle{} gender variable is missing in the data."}\NormalTok{)}
\NormalTok{\}}

\CommentTok{\# Display gender counts and verify they match total students}
\NormalTok{gender\_counts }\OtherTok{\textless{}{-}} \FunctionTok{table}\NormalTok{(pisa\_data}\SpecialCharTok{$}\NormalTok{gender)}
\FunctionTok{cat}\NormalTok{(}\StringTok{"Gender Counts:}\SpecialCharTok{\textbackslash{}n}\StringTok{"}\NormalTok{)}
\end{Highlighting}
\end{Shaded}

\begin{verbatim}
## Gender Counts:
\end{verbatim}

\begin{Shaded}
\begin{Highlighting}[]
\FunctionTok{print}\NormalTok{(gender\_counts)}
\end{Highlighting}
\end{Shaded}

\begin{verbatim}
## 
## Female   Male 
##   4451   4044
\end{verbatim}

\begin{Shaded}
\begin{Highlighting}[]
\NormalTok{total\_gender\_count }\OtherTok{\textless{}{-}} \FunctionTok{sum}\NormalTok{(gender\_counts)}
\NormalTok{total\_students }\OtherTok{\textless{}{-}} \FunctionTok{nrow}\NormalTok{(pisa\_data)}
\FunctionTok{cat}\NormalTok{(}\FunctionTok{sprintf}\NormalTok{(}\StringTok{"}\SpecialCharTok{\textbackslash{}n}\StringTok{Total students (gender counts): \%d}\SpecialCharTok{\textbackslash{}n}\StringTok{Total students in dataset: \%d}\SpecialCharTok{\textbackslash{}n}\StringTok{"}\NormalTok{, total\_gender\_count, total\_students))}
\end{Highlighting}
\end{Shaded}

\begin{verbatim}
## 
## Total students (gender counts): 8495
## Total students in dataset: 8495
\end{verbatim}

\begin{Shaded}
\begin{Highlighting}[]
\ControlFlowTok{if}\NormalTok{ (total\_gender\_count }\SpecialCharTok{==}\NormalTok{ total\_students) \{}
    \FunctionTok{cat}\NormalTok{(}\StringTok{"The gender data matches the total number of students.}\SpecialCharTok{\textbackslash{}n}\StringTok{"}\NormalTok{)}
\NormalTok{\} }\ControlFlowTok{else}\NormalTok{ \{}
    \FunctionTok{cat}\NormalTok{(}\StringTok{"Warning: Gender data does not match total students.}\SpecialCharTok{\textbackslash{}n}\StringTok{"}\NormalTok{)}
\NormalTok{\}}
\end{Highlighting}
\end{Shaded}

\begin{verbatim}
## The gender data matches the total number of students.
\end{verbatim}

After converting and counting the gender variable, the dataset
dimensions are displayed, followed by a preview of the first 10 rows.
These steps confirm the successful data import and provide an overview
of the dataset's structure, ensuring that all necessary variables and
observations are still present.

\begin{Shaded}
\begin{Highlighting}[]
\CommentTok{\# Get dimensions of the loaded data}
\NormalTok{data\_dim }\OtherTok{\textless{}{-}} \FunctionTok{dim}\NormalTok{(pisa\_data)}
\FunctionTok{cat}\NormalTok{(}\FunctionTok{sprintf}\NormalTok{(}\StringTok{"The \%s dataset contains \%d rows and \%d columns.}\SpecialCharTok{\textbackslash{}n}\StringTok{"}\NormalTok{, year, data\_dim[}\DecValTok{1}\NormalTok{], data\_dim[}\DecValTok{2}\NormalTok{]))}
\end{Highlighting}
\end{Shaded}

\begin{verbatim}
## The 2022 dataset contains 8495 rows and 1455 columns.
\end{verbatim}

\begin{Shaded}
\begin{Highlighting}[]
\CommentTok{\# Display a preview of the first 10 rows}
\FunctionTok{head}\NormalTok{(pisa\_data, }\DecValTok{10}\NormalTok{)}
\end{Highlighting}
\end{Shaded}

\begin{verbatim}
## # A tibble: 10 x 1,455
##    CNT   CNTRYID CNTSCHID CNTSTUID CYC   NatCen STRATUM SUBNATIO  OECD ADMINMODE
##    <chr>   <dbl>    <dbl>    <dbl> <chr> <chr>  <chr>      <dbl> <dbl>     <dbl>
##  1 THA       764 76400001 76400396 08MS  076400 THA97    7640000     0         2
##  2 THA       764 76400001 76400632 08MS  076400 THA97    7640000     0         2
##  3 THA       764 76400001 76400865 08MS  076400 THA97    7640000     0         2
##  4 THA       764 76400001 76400936 08MS  076400 THA97    7640000     0         2
##  5 THA       764 76400001 76401306 08MS  076400 THA97    7640000     0         2
##  6 THA       764 76400001 76401652 08MS  076400 THA97    7640000     0         2
##  7 THA       764 76400001 76401681 08MS  076400 THA97    7640000     0         2
##  8 THA       764 76400001 76401944 08MS  076400 THA97    7640000     0         2
##  9 THA       764 76400001 76402440 08MS  076400 THA97    7640000     0         2
## 10 THA       764 76400001 76402738 08MS  076400 THA97    7640000     0         2
## # i 1,445 more variables: Option_CT <lgl>, Option_FL <lgl>, Option_ICTQ <dbl>,
## #   Option_WBQ <dbl>, Option_PQ <dbl>, Option_TQ <dbl>, Option_UH <dbl>,
## #   BOOKID <dbl>, ST001D01T <dbl>, ST003D02T <dbl>, ST003D03T <dbl>,
## #   ST004D01T <dbl>, ST250Q01JA <dbl>, ST250Q02JA <dbl>, ST250Q03JA <dbl>,
## #   ST250Q04JA <dbl>, ST250Q05JA <dbl>, ST250D06JA <dbl>, ST250D07JA <dbl>,
## #   ST251Q01JA <dbl>, ST251Q02JA <dbl>, ST251Q03JA <dbl>, ST251Q04JA <dbl>,
## #   ST251Q06JA <dbl>, ST251Q07JA <dbl>, ST251D08JA <dbl>, ST251D09JA <dbl>, ...
\end{verbatim}

The PISA 2022 Thailand dataset provides a structured array of variables
categorized by prefixes to represent different dimensions of educational
data. Key identifiers include variables with the CNT prefix, such as
CNTRYID (Country Identifier), CNTSCHID (School ID), and CNTSTUID
(Student ID), which are essential for distinguishing observations by
country, school, and student.

Student responses (ST-prefixed variables) capture individual data, while
ICT-related variables (IC prefix) provide insights into technology
access and use. School questionnaire variables (SCH prefix) include
institutional characteristics and policies that may influence
educational outcomes. This foundational structure, combined with
supplementary data from additional PISA cycles and the National
Statistical Office, will enable a comprehensive examination of factors
influencing educational outcomes in Thailand.

\hypertarget{data-preparation-and-cleaning}{%
\subsection{Data Preparation and
Cleaning}\label{data-preparation-and-cleaning}}

Since this study focuses exclusively on Thailand, only Thailand-specific
data files provided by PISA Thailand were used, eliminating the need to
filter out data from other countries. This targeted dataset selection
simplifies storage requirements and streamlines file handling, allowing
for efficient data management across multiple PISA cycles (i.e., 2000,
2003, 2006, 2009, 2012, 2015, 2018, and 2022).

With the data imported and initial structure checks completed, the next
steps involve preparing the data for analysis. This includes handling
plausible values (PVs) and weights, addressing missing data, and
performing any necessary transformations to ensure a clean and
structured dataset.

\textbf{Handling of Plausible Values (PVs) and Weights}

The PISA dataset includes Plausible Values (PVs) for student achievement
scores in reading, mathematics, science, and creative thinking. PVs are
multiple imputed estimates of students' latent abilities rather than
exact scores, which means they are essential for accurate statistical
inferences and valid comparisons across student groups. Unlike single
point estimates, PVs account for measurement error by providing a range
of estimates, enhancing the validity of inferences drawn from these
scores. For this study, PVs will be applied in all analyses involving
student performance, aligning with best practices in large-scale
assessments to ensure robust statistical conclusions.

Additionally, survey weights are provided in the PISA dataset to adjust
for the complex sampling designs used, ensuring the sample accurately
represents the broader population. Since PISA employs stratified
sampling to capture diverse student populations, applying these weights
corrects for selection probability differences, allowing for
generalizable and unbiased parameter estimates. Weights are particularly
critical in maintaining statistical rigor in this study, as they ensure
that results are representative of Thailand's student population. In all
analyses involving PVs, the corresponding survey weights will be applied
to control for sampling biases and uphold the validity of statistical
inferences.

\textbf{Missing Data Handling}

Missing data is a common issue in large-scale assessments like PISA,
where students may skip questions or fail to complete the survey. Proper
handling of missing data is crucial to maintain the representativeness
of the sample and ensure the validity of the results. In this section,
various methods are employed to address missing data, following OECD
recommendations and best practices in educational research.

To ensure a thorough understanding of the missing data across multiple
years of PISA data, missing data summaries were generated for each
specified year using the coding below. For each variable, the percentage
of missing values was calculated, with results saved in separate CSV
files organized by year within the metadata/missing\_data\_summaries
folder. Each file is named according to the format
missing\_pisa\_data\_summary\_{[}year{]}.csv, providing a structured
overview of missing data without overloading this document with detailed
output. Readers interested in the complete summaries for each year can
refer to the CSV files in the scripts folder.

\begin{Shaded}
\begin{Highlighting}[]
\CommentTok{\# List of PISA years to process}
\NormalTok{years }\OtherTok{\textless{}{-}} \FunctionTok{c}\NormalTok{(}\DecValTok{2000}\NormalTok{, }\DecValTok{2003}\NormalTok{, }\DecValTok{2006}\NormalTok{, }\DecValTok{2009}\NormalTok{, }\DecValTok{2012}\NormalTok{, }\DecValTok{2015}\NormalTok{, }\DecValTok{2018}\NormalTok{, }\DecValTok{2022}\NormalTok{)}

\CommentTok{\# Function to dynamically load PISA data for a given year}
\NormalTok{load\_pisa\_data }\OtherTok{\textless{}{-}} \ControlFlowTok{function}\NormalTok{(year) \{}
\NormalTok{    csv\_path }\OtherTok{\textless{}{-}} \FunctionTok{paste0}\NormalTok{(}\StringTok{"data/raw/"}\NormalTok{, year, }\StringTok{"/pisa"}\NormalTok{, year, }\StringTok{"\_data.csv"}\NormalTok{)}
\NormalTok{    pisa\_data }\OtherTok{\textless{}{-}} \FunctionTok{read.csv}\NormalTok{(csv\_path)}
    \FunctionTok{return}\NormalTok{(pisa\_data)}
\NormalTok{\}}

\CommentTok{\# Loop through each year to calculate missing data summaries and save to metadata folder}
\ControlFlowTok{for}\NormalTok{ (year }\ControlFlowTok{in}\NormalTok{ years) \{}
    \CommentTok{\# Load data for the specific year}
\NormalTok{    pisa\_data }\OtherTok{\textless{}{-}} \FunctionTok{load\_pisa\_data}\NormalTok{(year)}
    
    \CommentTok{\# Calculate percentage of missing data for each variable}
\NormalTok{    missing\_data\_summary }\OtherTok{\textless{}{-}} \FunctionTok{sapply}\NormalTok{(pisa\_data, }\ControlFlowTok{function}\NormalTok{(x) }\FunctionTok{sum}\NormalTok{(}\FunctionTok{is.na}\NormalTok{(x)) }\SpecialCharTok{/} \FunctionTok{length}\NormalTok{(x) }\SpecialCharTok{*} \DecValTok{100}\NormalTok{)}
\NormalTok{    missing\_data\_summary }\OtherTok{\textless{}{-}} \FunctionTok{data.frame}\NormalTok{(}\AttributeTok{Variable =} \FunctionTok{names}\NormalTok{(missing\_data\_summary),}
                                       \AttributeTok{Missing\_Percentage =}\NormalTok{ missing\_data\_summary)}
    
    \CommentTok{\# Define year{-}specific folder path in metadata/missing\_data\_summaries}
\NormalTok{    year\_folder }\OtherTok{\textless{}{-}} \FunctionTok{paste0}\NormalTok{(}\StringTok{"data/metadata/missing\_data\_summaries/"}\NormalTok{, year)}
    \ControlFlowTok{if}\NormalTok{ (}\SpecialCharTok{!}\FunctionTok{dir.exists}\NormalTok{(year\_folder)) \{}
        \FunctionTok{dir.create}\NormalTok{(year\_folder, }\AttributeTok{recursive =} \ConstantTok{TRUE}\NormalTok{)}
\NormalTok{    \}}
    
    \CommentTok{\# Save missing data summary as a CSV file in the appropriate year subfolder}
\NormalTok{    summary\_file }\OtherTok{\textless{}{-}} \FunctionTok{paste0}\NormalTok{(year\_folder, }\StringTok{"/missing\_pisa\_data\_summary\_"}\NormalTok{, year, }\StringTok{".csv"}\NormalTok{)}
    \FunctionTok{write.csv}\NormalTok{(missing\_data\_summary, summary\_file, }\AttributeTok{row.names =} \ConstantTok{FALSE}\NormalTok{)}
\NormalTok{\}}
\end{Highlighting}
\end{Shaded}

The code below processes the PISA data for each specified year,
identifies and removes variables with 100\% missing data, and saves the
cleaned datasets in the pre-existing year-specific folders within
\texttt{data/processed/}. Each cleaned file is named
\texttt{cleaned\_pisa\_data\_{[}year{]}.csv}. This approach maintains an
organized structure for multi-year data analysis, storing each year's
cleaned data in its respective folder for easy reference and subsequent
analysis.

\begin{Shaded}
\begin{Highlighting}[]
\CommentTok{\# Load necessary libraries}
\FunctionTok{library}\NormalTok{(dplyr)}
\FunctionTok{library}\NormalTok{(readr)}

\CommentTok{\# Function to dynamically load the PISA data for a specified year}
\NormalTok{load\_pisa\_data }\OtherTok{\textless{}{-}} \ControlFlowTok{function}\NormalTok{(year) \{}
    \CommentTok{\# Define file path dynamically}
\NormalTok{    csv\_path }\OtherTok{\textless{}{-}} \FunctionTok{paste0}\NormalTok{(}\StringTok{"data/raw/"}\NormalTok{, year, }\StringTok{"/pisa"}\NormalTok{, year, }\StringTok{"\_data.csv"}\NormalTok{)}
    \CommentTok{\# Read the CSV}
\NormalTok{    pisa\_data }\OtherTok{\textless{}{-}} \FunctionTok{read\_csv}\NormalTok{(csv\_path)}
    \FunctionTok{return}\NormalTok{(pisa\_data)}
\NormalTok{\}}

\CommentTok{\# List of years you have PISA data for}
\NormalTok{years }\OtherTok{\textless{}{-}} \FunctionTok{c}\NormalTok{(}\DecValTok{2000}\NormalTok{, }\DecValTok{2003}\NormalTok{, }\DecValTok{2006}\NormalTok{, }\DecValTok{2009}\NormalTok{, }\DecValTok{2012}\NormalTok{, }\DecValTok{2015}\NormalTok{, }\DecValTok{2018}\NormalTok{, }\DecValTok{2022}\NormalTok{)}

\CommentTok{\# Loop through each year, load the data, clean it, and save the result}
\ControlFlowTok{for}\NormalTok{ (year }\ControlFlowTok{in}\NormalTok{ years) \{}
    \CommentTok{\# Load data for the specified year}
\NormalTok{    pisa\_data }\OtherTok{\textless{}{-}} \FunctionTok{load\_pisa\_data}\NormalTok{(year)}
    
    \CommentTok{\# Calculate the percentage of missing data for each variable}
\NormalTok{    missing\_data\_summary }\OtherTok{\textless{}{-}} \FunctionTok{sapply}\NormalTok{(pisa\_data, }\ControlFlowTok{function}\NormalTok{(x) }\FunctionTok{sum}\NormalTok{(}\FunctionTok{is.na}\NormalTok{(x)) }\SpecialCharTok{/} \FunctionTok{length}\NormalTok{(x) }\SpecialCharTok{*} \DecValTok{100}\NormalTok{)}
\NormalTok{    missing\_data\_summary }\OtherTok{\textless{}{-}} \FunctionTok{data.frame}\NormalTok{(}\AttributeTok{Variable =} \FunctionTok{names}\NormalTok{(missing\_data\_summary), }
                                       \AttributeTok{Missing\_Percentage =}\NormalTok{ missing\_data\_summary)}
    
    \CommentTok{\# Identify variables with 100\% missing data}
\NormalTok{    high\_missing\_summary }\OtherTok{\textless{}{-}}\NormalTok{ missing\_data\_summary }\SpecialCharTok{\%\textgreater{}\%}
      \FunctionTok{filter}\NormalTok{(Missing\_Percentage }\SpecialCharTok{==} \DecValTok{100}\NormalTok{)}
    
    \CommentTok{\# Remove variables with 100\% missing data}
\NormalTok{    cleaned\_data }\OtherTok{\textless{}{-}}\NormalTok{ pisa\_data }\SpecialCharTok{\%\textgreater{}\%} \FunctionTok{select}\NormalTok{(}\SpecialCharTok{{-}}\FunctionTok{one\_of}\NormalTok{(high\_missing\_summary}\SpecialCharTok{$}\NormalTok{Variable))}
    
    \CommentTok{\# Save the cleaned dataset in the appropriate processed year folder}
\NormalTok{    output\_path }\OtherTok{\textless{}{-}} \FunctionTok{paste0}\NormalTok{(}\StringTok{"data/processed/"}\NormalTok{, year, }\StringTok{"/pisa"}\NormalTok{, year, }\StringTok{"\_cleaned\_stage1.csv"}\NormalTok{)}
    \FunctionTok{write\_csv}\NormalTok{(cleaned\_data, output\_path)}
    
    \CommentTok{\# Print a message confirming the process for each year}
    \FunctionTok{cat}\NormalTok{(}\FunctionTok{sprintf}\NormalTok{(}\StringTok{"Year \%d: Removed \%d variables with 100\%\% missing data. Cleaned data saved to \%s}\SpecialCharTok{\textbackslash{}n}\StringTok{"}\NormalTok{, }
\NormalTok{                year, }\FunctionTok{nrow}\NormalTok{(high\_missing\_summary), output\_path))}
\NormalTok{\}}
\end{Highlighting}
\end{Shaded}

\begin{verbatim}
## Year 2000: Removed 0 variables with 100% missing data. Cleaned data saved to data/processed/2000/pisa2000_cleaned_stage1.csv
## Year 2003: Removed 0 variables with 100% missing data. Cleaned data saved to data/processed/2003/pisa2003_cleaned_stage1.csv
## Year 2006: Removed 0 variables with 100% missing data. Cleaned data saved to data/processed/2006/pisa2006_cleaned_stage1.csv
## Year 2009: Removed 0 variables with 100% missing data. Cleaned data saved to data/processed/2009/pisa2009_cleaned_stage1.csv
## Year 2012: Removed 0 variables with 100% missing data. Cleaned data saved to data/processed/2012/pisa2012_cleaned_stage1.csv
## Year 2015: Removed 1 variables with 100% missing data. Cleaned data saved to data/processed/2015/pisa2015_cleaned_stage1.csv
## Year 2018: Removed 20 variables with 100% missing data. Cleaned data saved to data/processed/2018/pisa2018_cleaned_stage1.csv
## Year 2022: Removed 137 variables with 100% missing data. Cleaned data saved to data/processed/2022/pisa2022_cleaned_stage1.csv
\end{verbatim}

For a complete list of variables with high missing percentages, please
refer to the supplementary file
\href{data/metadata/missing_data_summary.csv}{Missing Data Summary}.

\textbf{Variable Transformation}

Detail any modifications or transformations applied to the variables.
Provide the R code for these transformations.

\textbf{Descriptive Analysis}

Present basic statistics or visualizations to describe the data. Include
the R code generating these descriptive statistics.

\hypertarget{variable-selection}{%
\subsection{Variable Selection}\label{variable-selection}}

The selection of variables for analysis in this study was conducted with
a systematic approach rooted in the theoretical foundations as
stipulated by the PISA ICT Framework and the MLFTAU. Each chosen
variable reflects a deliberate intersection of these models, aiming to
dissect the intricate dynamics of ICT utilization within educational
domains both inside and beyond the classroom setting.

Variables were initially screened for their relevance to the core
constructs of the study's guiding frameworks. The PISA ICT Framework
provides a dichotomy of in-school factors (such as ICT curriculum
integration, availability of technology, and teacher ICT proficiency)
and outside-the-classroom influences (encompassing access to technology
at home and students' ICT engagement outside school hours). Similarly,
the Multi-Level Framework of Technology Acceptance and Use (MLFTAU)
offered a layered perspective on ICT adoption, spanning individual,
organizational, and environmental influences.

An extensive review of current literature was performed to identify
variables consistently linked with meaningful ICT integration outcomes.
These variables were cross-referenced with the factors outlined in the
theoretical frameworks to ensure their pertinence to the study's focus
on technology acceptance and educational effectiveness.

A meticulous examination of the PISA 2022 dataset was undertaken to
determine the availability of potential variables. This step was crucial
in refining the variable pool to include only those with robust data
support and exclude variables with extensive missing data or inadequate
measurement reliability.

The final assortment of variables was curated to ensure tangible
linkages to policy implications and educational practice. This was
underpinned by the consideration of variables that can inform
evidence-based interventions and strategic educational planning.

The selected variables encompass a comprehensive array of factors
relating to the PISA ICT framework and the MLFTAU:

\begin{longtable}[l]{>{\raggedright\arraybackslash}p{2cm}>{\raggedright\arraybackslash}p{3cm}>{\raggedright\arraybackslash}p{5cm}>{\raggedright\arraybackslash}p{5cm}}
\caption{\label{tab:in_school_constructs}Key Constructs of the PISA ICT Framework - In-School Dimension}\\
\toprule
Dimension & Construct & Definition & Relevance\\
\midrule
Context & Socioeconomic Status & The economic and social position of students and their families, influencing access to and use of ICT. & Provides background for understanding ICT access disparities.\\
 & School Location & The geographical location of the school (urban or rural) and its impact on ICT resources. & Helps identify regional differences in ICT integration.\\
 & Cultural Attitudes & Cultural beliefs and attitudes towards technology and education at school. & Offers insight into the social context of ICT use at school.\\
Input & ICT Resources & Availability and quality of technology and digital tools in educational settings at school. & Assesses the infrastructure for ICT integration at school.\\
 & Teacher Training in ICT & Extent of teacher preparation for integrating ICT into teaching practices at school. & Evaluates the readiness of educators for ICT use at school.\\
\addlinespace
 & School Policies on ICT & Policies and guidelines regarding ICT use in education set by schools or authorities. & Indicates the regulatory environment for ICT at school.\\
 & Student Engagement (Attitudes) & Students' perceptions, beliefs, and feelings towards using ICT in their learning environment at school. & Affects the willingness to engage with ICT and can significantly influence the acceptance and effective use of technology in education.\\
Process & ICT Use in Teaching & How teachers incorporate ICT into their teaching methods and curriculum at school. & Reflects the application of ICT in instructional practices at school.\\
 & Pedagogical Approaches with ICT & The use of ICT to support various teaching strategies, such as student-centered learning or collaborative learning. & Assesses how ICT is integrated into pedagogical practices to enhance teaching effectiveness.\\
 & Student Engagement (Time) & The amount of time that students spend using ICT for learning activities during school hours. & Indicates the level of integration of ICT in daily educational activities and can be a predictor of skill acquisition and comfort with technology.\\
\addlinespace
 & Student engagement (Attention) & The degree of students' focus or concentration when using ICT for learning purposes in school. & Reflects the quality of ICT engagement and may correlate with deeper learning and retention of information.\\
 & Student engagement (Effort) & The amount of energy or diligence that students apply to ICT-related tasks during school-based learning. & Suggests how motivated students are to use ICT resources for learning and may affect the outcomes of technology-enhanced education.\\
Output & Educational Outcomes & Impact of ICT on students' academic performance in areas such as reading, mathematics, and science at school. & Evaluates the effectiveness of ICT in enhancing learning at school.\\
 & ICT Competencies & Skills and abilities of students to effectively use ICT for educational purposes at school. & Assesses the development of digital literacy and competencies at school.\\
\bottomrule
\end{longtable}

\begin{longtable}[t]{>{\raggedright\arraybackslash}p{2cm}>{\raggedright\arraybackslash}p{3cm}>{\raggedright\arraybackslash}p{5cm}>{\raggedright\arraybackslash}p{5cm}}
\caption{\label{tab:out_of_school_constructs}Key Constructs of the PISA ICT Framework - Outside-the-Classroom Dimension}\\
\toprule
Dimension & Construct & Definition & Relevance\\
\midrule
Context & Socioeconomic Status & The economic and social position of students and their families, influencing access to and use of ICT. & Provides background for understanding ICT access disparities.\\
 & Cultural Attitudes & Cultural/family beliefs and attitudes towards technology and education out of school. & Offers insight into the social and family context of ICT use out of school.\\
Input & Access to ICT & Availability of ICT resources and tools for students outside the school environment. & Assesses opportunities for students to engage with ICT for learning beyond the school setting.\\
 & Parental/Teacher Supervision in Using ICT & The encouragement and resources provided by parents and teachers that facilitate the use of ICT for learning at home. & Assesses the home learning context, as parental and teacher involvement can significantly influence students' ability and motivation to use ICT for educational purposes outside of school.\\
Process & ICT Use for Learning (Homework-Related Activities) & Extent of student use of ICT for homework-related learning activities outside the classroom. & Measures student engagement with ICT for structured educational purposes out of school.\\
\addlinespace
 & ICT Use for Learning (Self-Directed Learning Activities) & Extent of student use of ICT for self-directed learning activities outside the classroom. & Measures student engagement with ICT for unstructured educational purposes out of school.\\
 & ICT Use for Leisure & Extent of student use of ICT for leisure activities outside the classroom. & Measures student engagement with ICT for leisure purposes out of school.\\
Output & Educational Outcomes & Impact of ICT on students' academic performance in areas such as reading, mathematics, and science out of school. & Evaluates the effectiveness of ICT in enhancing learning out of school.\\
 & Student Well-being & Influence of ICT use on students' emotional and psychological well-being out of school. & Examines the broader effects of ICT on student welfare out of school.\\
\bottomrule
\end{longtable}

\begin{longtable}[t]{>{\raggedright\arraybackslash}p{2cm}>{\raggedright\arraybackslash}p{3cm}>{\raggedright\arraybackslash}p{5cm}>{\raggedright\arraybackslash}p{5cm}}
\caption{\label{tab:MLFTAU_constructs}Key Constructs of the Multi-Level Framework of Technology Acceptance and Use (MLFTAU)}\\
\toprule
Dimension & Construct & Definition & Relevance\\
\midrule
Individual & Performance Expectancy (PE) & The degree to which an individual believes that using the technology will help them to attain gains in job performance. & Assesses how students perceive the use of ICT in improving their learning outcomes.\\
 & Effort Expectancy (EE) & The degree of ease associated with the use of the technology. & Assesses how the perceived ease of using ICT impacts students' willingness to engage with it for educational purposes.\\
 & Social Influence (SI) & The degree to which an individual perceives that important others believe they should use the new technology. & Assesses the impact of peer and parental influence on students' use of ICT in education.\\
 & Facilitating Conditions (FC) & The degree to which an individual believes that an organisational and technical infrastructure exists to support the use of the technology. & Assesses the availability of ICT resources and support in schools that enable students to use ICT effectively.\\
 & Hedonic Motivation & The pleasure or pain derived from using the technology. & Assesses how the enjoyment of using ICT influences students' engagement with technology for learning.\\
\addlinespace
 & Price Value & The individual's cognitive trade-off between the perceived benefits of the technology and its monetary cost. & Although this construct may be more relevant in consumer settings, it could be adapted to assess the perceived value of investing time and effort in using ICT for educational purposes.\\
 & Habit & The extent to which individuals tend to perform behaviours automatically because of learning. & Assesses how habitual use of ICT outside of school settings influences students' use of ICT for learning.\\
Group & Group Norms & The shared expectations about behaviour within the group. & Assesses how classroom or school norms regarding ICT use impact students' attitudes and behaviours.\\
 & Group Cohesion & The extent to which group members stick together and remain united in the pursuit of common goals. & Assesses how collaborative learning environments using ICT influence educational outcomes.\\
 & Team Climate for Innovation & The shared perceptions of the group's practices, procedures, and behaviours that support innovation. & Assesses how a school's climate for innovation affects the adoption and effective use of ICT in education.\\
\addlinespace
Organisation & Organisational Facilitating Conditions & The degree to which an organisation provides the infrastructure and support for the use of the technology. & Assesses how the school's ICT infrastructure and policies support or hinder the integration of ICT in education.\\
 & Organisational Culture & The shared values, beliefs, and practices within an organisation that shape behaviour and attitudes toward technology use. & Assesses the school's culture regarding how technology can provide insights into the systemic factors influencing ICT integration in education.\\
 & Leadership Support & The extent to which leaders within the organisation encourage and support the use of technology. & Assesses the role of school leadership in promoting or impeding the use of ICT for educational purposes.\\
Cross-Level Constructs & Cross-Level Interactions & The interactions between constructs at different levels that influence technology acceptance and use. & Assesses how individual attitudes interact with group norms or organisational culture to impact ICT use in educational settings.\\
Outcomes (Output) & Educational Outcomes & Impact of ICT on students' academic performance in areas such as reading, mathematics, and science out of school. & Evaluates the effectiveness of ICT in enhancing learning out of school.\\
\addlinespace
 & Student Well-being & Influence of ICT use on students' emotional and psychological well-being out of school. & Examines the broader effects of ICT on student welfare out of school.\\
\bottomrule
\end{longtable}

The following table provides an example of how In-School Input variables
are mapped to the research objectives. By focusing on key resources and
conditions within schools---such as ICT availability, teacher training,
and internet connectivity---this mapping demonstrates how each variable
supports an in-depth understanding of ICT integration at the school
level. This structured selection ensures that each variable is both
theoretically grounded in the PISA ICT framework and practically
relevant to evaluating in-school ICT infrastructure's role in
educational outcomes.

\begin{longtable}[t]{>{\raggedright\arraybackslash}p{2cm}>{\raggedright\arraybackslash}p{2cm}>{\raggedright\arraybackslash}p{3cm}>{\raggedright\arraybackslash}p{1.5cm}>{\raggedright\arraybackslash}p{6cm}}
\caption{\label{tab:example_mapping_of_in-school_input_variables}In-School ICT Input Variables: Alignment with Research Objectives on ICT Integration in Education}\\
\toprule
Framework Construct & Variable & Description & Data Source & Relevance to Research Objectives\\
\midrule
Input (In-School) & ICT Availability & Number and quality of computers and digital devices available in classrooms & PISA 2022 & Assesses the basic ICT infrastructure in schools, crucial for equitable access and effective ICT integration\\
 & Teacher ICT Training & Level of teacher training or professional development in ICT skills & PISA 2022 & Evaluates teacher readiness to implement ICT in teaching, influencing student engagement with technology\\
 & Digital Learning Resources & Access to digital resources like educational software and online tools in schools & PISA 2022 & Measures the variety of resources available to support diverse ICT-based teaching methods\\
 & Internet Connectivity & Quality and speed of internet access within the school environment & PISA 2022 & Determines the school’s capacity to support internet-based learning, a critical component of ICT integration\\
 & Technical Support for ICT & Availability of dedicated technical support staff for ICT maintenance & PISA 2022 & Provides insight into the ongoing support for ICT, which affects sustainability and reliability of ICT resources\\
\bottomrule
\end{longtable}

This systematic and rigorous variable selection protocol ensures that my
analysis is anchored in empirical evidence, is methodologically sound,
and aligns seamlessly with my research aims.

\hypertarget{analytical-methods}{%
\subsection{Analytical Methods}\label{analytical-methods}}

In the example analysis below, the plausible values and weight for the
specified year are accessed from the lists, and the average mathematics
performance is calculated by gender, demonstrating the practical use of
the data and lists.

\hypertarget{load-necessary-library}{%
\section{Load necessary library}\label{load-necessary-library}}

library(intsvy)

\hypertarget{function-to-load-cleaned-pisa-data-for-a-specified-year}{%
\section{Function to load cleaned PISA data for a specified
year}\label{function-to-load-cleaned-pisa-data-for-a-specified-year}}

load\_cleaned\_pisa\_data \textless- function(year) \{ \# Define file
path dynamically for the processed folder with the correct naming format
csv\_path \textless- paste0(``data/processed/'', year, ``/pisa'', year,
``\_cleaned.csv'') \# Read the CSV pisa\_data \textless-
read\_csv(csv\_path) return(pisa\_data) \}

\hypertarget{specify-the-year-of-analysis}{%
\section{Specify the year of
analysis}\label{specify-the-year-of-analysis}}

year \textless- ``2022''

\hypertarget{load-the-cleaned-pisa-data-for-the-specified-year}{%
\section{Load the cleaned PISA data for the specified
year}\label{load-the-cleaned-pisa-data-for-the-specified-year}}

pisa\_data \textless- load\_cleaned\_pisa\_data(year)

\hypertarget{access-pvs-and-weight-variable-for-the-specified-year}{%
\section{Access PVs and weight variable for the specified
year}\label{access-pvs-and-weight-variable-for-the-specified-year}}

pvs \textless- pv\_list{[}{[}year{]}{]} weight\_var \textless-
weights{[}{[}year{]}{]}

\hypertarget{ensure-gender-is-properly-set-as-a-factor-in-the-data}{%
\section{Ensure `gender' is properly set as a factor in the
data}\label{ensure-gender-is-properly-set-as-a-factor-in-the-data}}

if (``gender'' \%in\% colnames(pisa\_data)) \{
pisa\_data\(gender <- factor(pisa_data\)gender, levels = c(``Female'',
``Male'')) \} else \{ stop(``The `gender' variable is missing in the
data.'') \}

\hypertarget{example-analysis-calculate-average-mathematics-performance-by-gender}{%
\section{Example analysis: Calculate average mathematics performance by
gender}\label{example-analysis-calculate-average-mathematics-performance-by-gender}}

math\_results \textless- pisa.mean.pv( pvlabel = pvs, by = ``gender'',
weight = weight\_var, data = pisa\_data )

\hypertarget{display-results}{%
\section{Display results}\label{display-results}}

print(math\_results)

\begin{verbatim}

The example below calculates the average mathematics performance by gender, using PVs and the appropriate student weight to ensure accuracy and representativeness.




# Load necessary library
library(intsvy)

# Define plausible values and weight variable
pvs_math <- c("PV1MATH", "PV2MATH", "PV3MATH", "PV4MATH", "PV5MATH")
weight_var <- "W_FSTUWT"

# Example analysis using plausible values and weights
# Average mathematics performance by gender
math_results <- pisa.mean.pv(pisa_data, pvlabel = pvs_math, by = "gender")

# Display results
print(math_results)
\end{verbatim}

A range of statistical techniques will be applied to analyze the data.

Descriptive Statistics: To provide an overview of key variables related
to ICT use and student performance.

Hierarchical Linear Modeling (HLM): This technique will be used to
account for the nested structure of the PISA data (students within
schools). HLM models will help analyze the relationships between ICT
access, school characteristics, and student performance.

Weighted Linear Regression: This will assess the impact of ICT use on
student outcomes while accounting for the survey's sampling design.

Plausible Values (PVs): PVs will be used to represent student
performance across multiple dimensions, accounting for uncertainty in
test scores. The EdSurvey package will be utilized to handle PVs in
analysis.

\hypertarget{model-validation}{%
\subsection{Model Validation}\label{model-validation}}

Validation techniques, such as cross-validation and sensitivity
analysis, will be employed to ensure the robustness of the models. This
includes testing for assumptions, identifying potential biases, and
assessing model stability.

\hypertarget{addressing-validity-and-reliability}{%
\section{Addressing Validity and
Reliability}\label{addressing-validity-and-reliability}}

The study ensures both validity and reliability through carefully
designed data handling procedures, the use of well-established data
sources, and robust analytical techniques. Several strategies are
employed to maintain the highest standards of academic rigor:

\hypertarget{validity}{%
\subsection{Validity}\label{validity}}

Content Validity: The selection of variables is closely aligned with the
CIPO model and the Multi-Level Framework of Technology Acceptance and
Use (MLFTAU), ensuring that the constructs being measured accurately
reflect the key dimensions of ICT integration and educational outcomes.

Construct Validity: All key variables, such as ICT access, ICT
utilization, and student performance, are derived from reliable and
validated instruments within the PISA datasets and national databases,
providing confidence that these constructs are measured as intended.

Internal Validity: Potential confounding variables (e.g., socio-economic
status, school infrastructure) are controlled for using appropriate
statistical techniques such as Hierarchical Linear Modeling (HLM) and
weighted regression, minimizing the impact of extraneous factors on the
results.

External Validity: The use of national-level PISA data and a
well-defined sampling framework ensures that findings are generalizable
to the broader population of Thai students and schools, as well as
potentially applicable to other contexts with similar educational
environments.

\hypertarget{reliability}{%
\subsection{Reliability}\label{reliability}}

Measurement Reliability: The PISA dataset, along with other national
data sources, undergoes thorough validation and testing to ensure
consistency and accuracy in the measurement of variables. The use of
Plausible Values (PVs) further strengthens reliability by accounting for
the uncertainty inherent in student performance data.

Reproducibility: All steps of the analysis are fully documented in this
RMarkdown document, ensuring that the code, data processing steps, and
statistical analyses are transparent and reproducible. This includes the
use of the EdSurvey package to manage the complex sampling design and
PVs, ensuring that future researchers can replicate the analysis.

Analytical Techniques: Established statistical methods (e.g., HLM,
weighted linear regression) are applied rigorously, with models
validated through techniques such as cross-validation and sensitivity
analysis to check for consistency and robustness.

\hypertarget{documentation-and-transparency}{%
\subsection{Documentation and
Transparency}\label{documentation-and-transparency}}

All R code, data manipulations, and analysis steps are documented in
this RMarkdown file, ensuring that the research process is transparent
and easily replicable by others. The document adheres to best practices
for reproducible research, including version control through GitHub,
ensuring that any future updates or revisions are traceable.

By maintaining strict adherence to these principles, the study produces
results that are both accurate and generalizable, with confidence in the
reliability and validity of the findings. The combination of robust data
sources, appropriate statistical techniques, and clear documentation
ensures the integrity and reproducibility of the research.

\hypertarget{ethical-considerations}{%
\section{Ethical Considerations}\label{ethical-considerations}}

This research adheres to the highest standards of ethical conduct as
outlined by international and institutional guidelines for research
involving human data. Several key ethical principles guide the study to
ensure respect for individuals, data confidentiality, and the broader
sociocultural context:

\hypertarget{data-confidentiality-and-privacy}{%
\subsection{Data Confidentiality and
Privacy}\label{data-confidentiality-and-privacy}}

The data used in this study, including individual-level PISA data, are
anonymized to protect the identity and privacy of students and schools.
No personally identifiable information (PII) is used in the analysis,
and data are handled in accordance with national and international data
protection regulations (e.g., GDPR where applicable).

Strict access control measures are in place to ensure that the data is
only accessed by authorized personnel for the purposes of this research.
The dataset is stored securely, and all analyses are conducted in a
manner that prevents the identification of individual respondents.

The use of Plausible Values (PVs) in the PISA dataset enhances privacy
by ensuring that no single score can be directly attributed to a
specific student, thereby preserving confidentiality.

\hypertarget{transparency-and-honesty-in-reporting}{%
\subsection{Transparency and Honesty in
Reporting}\label{transparency-and-honesty-in-reporting}}

All findings are reported accurately and without manipulation or bias.
The methodology is fully transparent, allowing for the reproducibility
and scrutiny of results. Potential limitations or biases in the data are
acknowledged and addressed, ensuring that the findings contribute
meaningfully to both academic knowledge and policy discussions.

In line with best practices, the study commits to sharing anonymized
data, code, and methodology transparently via platforms such as GitHub,
allowing for peer review and replicability of the research.

\hypertarget{respect-for-sociocultural-context}{%
\subsection{Respect for Sociocultural
Context}\label{respect-for-sociocultural-context}}

The study recognizes the importance of Thailand's unique sociocultural
and educational landscape. Special attention is paid to ensuring that
the research does not reinforce inequities or biases inherent in the
data, particularly with respect to disadvantaged and marginalized
groups.

Ethical considerations include the contextual sensitivity of findings,
ensuring that policy recommendations are culturally appropriate and do
not disproportionately disadvantage any group. The study aims to
contribute positively to the development of Thailand's education system
by offering evidence-based, inclusive policy suggestions.

\hypertarget{compliance-with-ethical-review-boards}{%
\subsection{Compliance with Ethical Review
Boards}\label{compliance-with-ethical-review-boards}}

The study has been reviewed and approved by the Internal Review Board
(IRB) of the researcher's institution, ensuring that it meets ethical
standards regarding data use, privacy, and the handling of sensitive
information.

In line with institutional and international ethical guidelines (e.g.,
APA and the Declaration of Helsinki), the research adheres to principles
of respect for persons, beneficence, and justice, ensuring that all
steps in the research process protect the dignity and rights of
individuals represented in the data.

\hypertarget{policy-and-societal-implications}{%
\subsection{Policy and Societal
Implications:}\label{policy-and-societal-implications}}

The study is designed with careful consideration of its potential policy
implications. Recommendations made as a result of the findings aim to
improve educational outcomes in Thailand and to close the digital
divide, particularly for marginalized communities. Ethical
responsibility is taken to ensure that these recommendations are
grounded in evidence and are not used in ways that could negatively
impact vulnerable populations.

By adhering to these ethical principles, the research not only ensures
compliance with data protection and privacy standards but also
contributes positively to educational policy and practice in Thailand in
a responsible and culturally sensitive manner.

\end{document}
